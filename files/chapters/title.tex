\title{Angry Birds In Real Life: Et legetøj}
\author{Eksamensprojekt  \\ Fag: Teknik A Design og Produktion \\ Klasse: Design og produktion TK 1 \\ H.C. Ørsted Gymnasiet, Lyngby \\ Udarbejdet af: \\\noindent\begin{tabular}{ll}
\\[3ex]
\makebox[2.5in]{\hrulefill} & \makebox[2.5in]{\hrulefill}\\
Elev: Simon Rumle Tarnow & Dato\\[8ex]% adds space between the two sets of signatures
\makebox[2.5in]{\hrulefill} & \makebox[2.5in]{\hrulefill}\\
Elev: Daniel Muff Laporte & Dato\\[8ex]
\makebox[2.5in]{\hrulefill} & \makebox[2.5in]{\hrulefill}\\
Elev:\\ Christoffer Irvall Rasmussen & Dato\\[4ex]
\end{tabular}}
\maketitle
% \begin{center}
%  	\includegraphics[height=10cm]{figures/titlepicheart.png}
%  \end{center}
 \newpage
 \section*{Resume}
 Vi har i forbindelse med vores eksamensprojekt i Teknik A (Design og Produktion) udarbejdet et legetøj kaldet Angry Birds In Real Life, der drager inspiration fra Angry Birds spillet udviklet af. I denne rapport dokumenterer vi produktudviklingen med hovedfokus på El-tekniske aspekter og programmering. 

 Vi har i forbindelse med vores undersøgelse af problemstilling kommet frem til at mange unge spiller meget videospil, dette forsøger vi at hjælpe på ved at gøre videospillet mere interactivt og derved socialt, så det således er en lille "angry bird" kanon. Af anvendt teknik benyttes der motorer til gearing og retningsbestemmelse af kanonen, farvesensor benyttes til at give forskellige farve projektiler forskellige hastigheder, samt programmering af Arduino. Vi kom frem til at de tekniske aspekter af vores produkt fungerede fint, men vi har en noget mangelfuld protype i og med at mange af delene ikke er sat fast til selve kanonen.
 \newpage
\tableofcontents
\newpage

% Angry Birds In Real Life: Et legetøj


% Forside

% \begin{titlepage}
%     \centering
%     % \includegraphics[width=0.15\textwidth]{example-image-1x1}\par\vspace{1cm}
%     {\scshape\LARGE H. C. Ørsted Gymnasiet, Lyngby \par}
%     \vspace{1cm}
%     {\scshape\Large Forløb: Digitalt måleinstrument\par}
%     \vspace{1.5cm}
%     {\huge\bfseries \par}
%     \vspace{2cm}
%     {\Large\itshape Simon Rumle, Daniel Laporte og Christoffer Rasmussen\par}
%     \vfill
%     \par
%      \textsc{Klasse: TK 1, Fag: Teknik-Design og produktion}

%     \vfill

% % Bottom of the page
%     {\large \today\par}
% \end{titlepage}