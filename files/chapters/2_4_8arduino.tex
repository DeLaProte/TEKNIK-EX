\section{Arduinoen} \label{sec:arduino}
%--------------- Indsæt en arduino ---------------%
\begin{figure}[H]
	\centering
    \includegraphics[width=\textwidth]{figures/arduino/portmani.PNG}
	\caption{Arduino uno med ATmega328}
	\label{fig:portmani}
	Her kan man se hvilke ben der er på Arduinoen, og hvilke porte de fordelt på ( PORTB, PORTC og PORTD). Hvis man ser på "Port pin", står der P for port, efterfulgt af bogstavet for hvilken port den er tildelt, efterfulgt af dens bit position i den port.\\
	Vi har også opgivet den påkrævede spænding for at køre Arduinoen. Vi kan også se Analog input ben (A0-A5) og de digital input og output ben (0-13). Vi kan se ud fra analog ben A4 og A5 at de er SDA og SCL, som er relevant når vi skal benytte I2C kredsen.
	\\Kilde: \url{http://pighixxx.com/unov3pdf.pdf}
\end{figure}
\subsection{Analog inputs og outputs (PWM)}\label{sec:ard:analog}
Analog input i Arduinoen er et input som beskriver spændingsfaldet over inputtet og jord med et 10 bit tal.\cite{arduinoAnalog} Dette betyder at den laveste værdi og højeste værdi af et analog input er
\begin{align}
\SI{}{U_{MIN}}&=b00\,0000\,0000\rightarrow 0\rightarrow \SI{0}{V}\\
\SI{}{U_{MAX}}&=b11\,1111\,1111\rightarrow 1023\rightarrow \SI{5}{V}\\
\end{align}
Med Arduinoen kan man få et analog input igennem en analog ben (se \ref{fig:portmani}), og kalde metoden \emph{analogRead()} til benets nummer, eksempelvis A0. Da vores precision er på 10 bit, til at beskrive et maks spændingsfald på $\SI{5}{V}$, må det gælde at den laveste ændring i spændingsfaldet vi kan måle er
\begin{align}
U_{prec}&=\frac{\SI{5}{V}}{1023}\approx \SI{5d-3}{V}
\end{align}
\subsection{Digital inputs og outputs}
\subsection{I2C - Synkroniseret kommunikation}