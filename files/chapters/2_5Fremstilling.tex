\section{Fremstilling}

\subsection{Fumlebræt-modeller}
	\begin{figure}[H]
		\centering
	    \includegraphics[width=13cm]{figures/stock.jpg}
		\caption{}
		\label{fig:}
	\end{figure}
\subsection{PCB - fremstilling}\label{subs:pcbfremstilling}
	\begin{figure}[H]
		\centering
	    \includegraphics[width=13cm]{figures/stock.jpg}
		\caption{Et billede af et PCB vi designede i Livewire}
		\label{fig:PCBPrint}
	\end{figure}
	\begin{figure}[H]
		\centering
	    \includegraphics[width=13cm]{figures/stock.jpg}
		\caption{Et billede af et PCB fremstillet fysisk}
		\label{fig:PCBfysisk}
	\end{figure}
\subsection{Endelige prototype}\label{subs:endeligProto}
\begin{figure}[H]
	\centering
    \includegraphics[width=13cm]{figures/stock.jpg}
	\caption{Billede af endelig produkt}
	\label{fig:endeligPrototype}
\end{figure}
\todo{Beskriv hvordan PCB-baseret produkt ikke fungrede, har vi nogen mulige forklaringer?}
% ----------- MULIGE FORKLARINGER
% punkt 1 PCB'et havde muligvist ikke ordenlige forbindelser
% Dette har vi testet ved at bruge et multimeter se om der var gode forbindelser mellem alle punkter på PCB'et
% punkt 2 Der er mulighed for at da PCB'et manglede en ledning eller 2 da det blev lavet på computeren.
% Dette er noget som vi har prøvet at undgå ved at lave så få manuelle ændringer i PCB-wizard så der ikke blev overset nogle forbindelser. 
% punkt 3 Der var komponeter som ikke fungerede ordenligt.
% Det er ikke særlig ofte at komponeter ikke fungere ordenligt når vi bruger dem derfor kan det være at der er en komponent der er blev sat forkert i eller at der er blevet brugt et forkert komponent hvilket også er en del af dette punkt.
\subsection{Forbindelses testing}


