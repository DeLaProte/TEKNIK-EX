\section{Slut afprøvning} \label{sluttest}
Vi prøvede at få vores elektronik til at virke på den reelle kanon. Men det viser sig at vores kanon rør var for stort til at vores farve sensor kunne aflæse boldens farve.
Derfor har vi fået vores produkt til at virke uden for kanon røret.Vores produkt kan se om der er en bold tilstede med sin infrarøde sensor og herefter aflæse farven af bolden. farve sensoren har en tendens til nogle gange ikke at aflæse det rigtige hver gang, men dens målte resultat er rigtigt det meste af tiden.
Denne information er blevet successfult sendt til den anden arduino som har skrevet det på displayet

\todo{billede af display som viser hvilke farve det er}

på samme måde har vi også testet controlleren og motorcontrollen ved at se om det virker i realiteten. og det viser at kanon bevæger sig når der bliver trykket på den tilsvarede knap. Hvilket betyder at vores controller-, steppermotor, og DCkreds fungere korrekt.
Vores kode kører vores produkt korrekt da alle dele af koden er blevet testet og kører korrekt. Der viser sig dog at være det problem at arduinoerne har en tendens til at genstarte midt kørsel. Vi mener at det er strøm tilførelsen der er skyld i dette problem.

Med hensyn til kravspecificationerne så overholder vores produkt de fleste.

\begin{itemize}
\item Sikkert at bruge for børn.
\item Projektilerne skal ikke have en farlig form.
\item Produktet skal skyde hårdt nok til at vælte små papbokse.
\item Produktet skal ikke være i strid med våbenloven.
\end{itemize}
 
Her er det kun kravet om at vores produkt skal skyde hårdt nok til at vælte papbokse. Men hvis vi får lavet nogle forbedringer Så ville dette krav også blive opfyldt. 


\todo{Få skrevet om hvorfor bolden ikke passede}

% HAstighedsmåler 10.5 cm. Rør D = 6.5cm
% Rør 53 lang.
