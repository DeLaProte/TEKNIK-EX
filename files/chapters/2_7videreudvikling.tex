\section{Videreudvikling}
Der er to typer af forbedringer der som skal til før man har et endeligt produkt. Først skal alle de dele som voros prototype burde kunne klare laves. Bagefter kan man begynde at lave produktet mere brugervendeligt.\\

For at få prototypen til at virke skal selve kanon designet laves om. Dette kan gøres ved at tage den mekanisme der skal hive i elastikkerne og placere dem mere direkte bag ved kanonrøret for at undgå for mange unødige kræftpåvirkninger på resten af kanonen.\\
Udover dette er det nødtvendigt at bruge et mindre rør der har en størrelse der svarer til størelsen på bolden således at farvesensoren ikke opfanger farver fra selve røret og omgivelserne. Hertil ville det også være smart at male indersiden af røret sort.\\
Det ville sandsynligvis være en god ide at lave produktet af metal og 3D printet plastik ved et fremtidigt produkt da det giver mulighed for at producere lagt mere specialiserede dele. \\
For sikkerhed er det en god ide at finde en løsning til at produktet har mulighed for at gå i stykker hvis man vinkler kanonen for kraftigt. Dette kunne blive gjort ved at bruge gear som har en maksimum drejningsmoment som de kan klare før de begynger at skippe nogle tænder. Eller man kunne have nogle flere knapper for at se om kanonen har flyttet sig helt til en side.\\
Udover dette kunne man lave controlleren både mere stabil ved at lave et PCB og lave en ramme til den så man holde den behageligt i hånden.
