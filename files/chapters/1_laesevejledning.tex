\section{Læsevejledning}
\subsection{Afsnit}
Overordnede afsnit inklusiv produkt-blokke-afsnit gives et nummer højere end tidligere overordnede afsnit. Alle underafsnit markeres med det overordnede nummer til det afsnit det tilhører med en punktum-indeksering som fungerer hierarkisk. Så f.eks. afsnit "Projektbeskrivelse" vil blive markeret med "2". Hvor deraf et underafsnit som "Problemanalyse" vil blive markeret "2.1". Afsnit der er sideordnede med "Problemanalyse" og underordnede med "Projektbeskrivelse" vil så blive markeret "2.2", "2.3", "2.4" etcetera.  
\subsection{Bilag}
Bilag er til forskel fra normale afsnit i rapporten angivet med et stort bogstav i titlen, således at hvis det skrives "Se i bilag A" vides det at man skal se sidst i vores indholdsfortegnelse hvor der er et bilag indikeret med A i begyndelse. I PDF-format kan man trykke på references for at blive taget direkte til bilaget.
Jo større i omfang bilagene er desto senere i rapporten placeres de. Så blandt vores sidste bilag er programkoden til arduinoen og vores logbog.
\subsection{Kilder}
Kilderne er nummereret efter alfabetisk rækkefølge. Kilder er refereret til i teksten ved at de er nummereret og med kantede parenteser som f.eks. [1]. Kilderne kan klikkes på i PDF-format for at blive taget til kilden direkte til litteraturlisten. Der refereres forskelligt afhængig af hvilken type kilde det er:


\subsubsection{Websider}
Først indikeres firmaet eller forfatteren af kilden, dernæst overskriften på afsnittet kilden har. Derefter er der et hyperlink til URL med angivelse af dato sidst set. Til websider, som ikke er links til datablade, gives der kort kildekritik.


[Forfattere]. [Overskift]. URL: [link] ([dato sidst set]). [Kildekritik]

\subsubsection{Bøger og skrifter}
For bøger og andre skrifter. Indikeres først forfattere, derefter titlen på værket, efterfulgt af forlagets adresse og forlaget. Der knyttes evt. noter til værket så som hvilke sider vi har benyttet.

[Forfattere]. [Titel]. [Forlag adresse]: [Forlag]. [Noter]
